\documentclass{article}
\usepackage[utf8]{inputenc}
\usepackage[margin= 0.75 in]{geometry}
\usepackage{hyperref}
\usepackage{minted}
\usepackage{amsfonts}
\usepackage{amsmath}
\usepackage{amssymb}
\usepackage{tikz}
\usepackage{graphicx}
\usepackage{stmaryrd}
\DeclareMathOperator{\R}{\mathbb{R}}
\DeclareMathOperator{\C}{\mathbb{C}}
\DeclareMathOperator{\Z}{\mathbb{Z}}
\DeclareMathOperator{\N}{\mathbb{N}}
\DeclareMathOperator{\K}{\mathbb{K}}
\DeclareMathOperator{\Q}{\mathbb{Q}}
\DeclareMathOperator{\U}{\mathbb{U}}
\renewcommand{\k}{\llbracket 1, k \rrbracket}
\newcommand{\n}{\llbracket 1, n \rrbracket}


\title{Exercices de colles}
\date{2022-2023}
\author{Simon Queric}
\begin{document}
\maketitle

Il s'agit d'un polycopié d'exercices de mathématiques de première année de licence ou de prépa. Ce sont les exercices que je donne en colle chaque semaine à la MPSI1 du Lycée Charlemagne à Paris. 

TODO : rajouter des étoiles selon la difficulté de l'exercice. 

\clearpage
\tableofcontents
\clearpage

% \addcontentsline{toc} 

\large 
\section{Logique, rédaction et applications}

\subsection*{Questions de cours}
\setlength{\parindent}{0cm}

\begin{enumerate}
    \item Rappeler les relations coefficients/racines d'un polynôme de degré 2. 
    \item Soit une fonction $f:E\to F$. Montrer que pour tout $A, B \subseteq E$ on a $f(A\cup B) = f(A)\cup f(B)$ et pour tout $C, D \ f^{-1}(C\cup D) = f^{-1}(C)\cup f^{-1}(D)$ 
    \item Soient deux fonctions $f:E \to F$ et $g:F\to G$. Montrer que : 
   \begin{enumerate}
       \item $g\circ f$ injective $\Rightarrow$ $f$ injective 
       \item $g\circ f$ surjective $\Rightarrow$ $g$ surjective 
   \end{enumerate} 

\end{enumerate}


\subsection*{Exercices}
\setlength{\parindent}{0cm}

\subsection*{Exercice 1} 

Soit $f: x\in \R \mapsto x^2 \in \R$ et $A = [-1, 4]$. Déterminer $f(A), f^{-1}(A)$. Déterminer $\sin([0, 2\pi]), \sin([0, \pi/2]), \sin^{-1}([3, 4]), \sin^{-1}([-2, -1])$ 

\subsection*{Exercice 2} 

\begin{enumerate}
    \item Trouver un ensemble $E$, une partie $F \subsetneq E$ et une fonction $f : F \to E$ bijective. 
    \item Trouver une fonction $f:\N \to \N$ telle que tout entier $n$ possède une infinité d'antécédants par $f$. 
\end{enumerate}

\subsection*{Exercice 3} 

Soient $A, B, C \subseteq E$. Montrer que $A\cap B = A\cap C \Leftrightarrow A\cap B^c = A\cap C^c$ 

\subsection*{Exercice 4} 

Montrer que $(A\cup B) \cap C \subseteq A \cup (B\cap C)$ avec égalité ssi $A \subseteq C$. 

\subsection*{Exercice 5} 

Soit une fonction $f:E\to F$. Montrer que :
\begin{enumerate}
    \item $\forall A\subseteq F \ f(f^{-1}(A)) \subseteq A $ 
    \item $\forall B\subseteq E \ B \subseteq f^{-1}(f(B))$ 
\end{enumerate}

\subsection*{Exercice 6} 

\begin{enumerate}
    \item Soit $\varphi : X \in \mathcal{P}(\N) \mapsto X\cap 2\N \in \mathcal{P}(\N)$. $\varphi$ est-elle injective ? surjective ? quelle est son image ?
    \item Soit $E$ un ensemble non vide, $A\subseteq E$ et $\phi : X \in \mathcal{P}(E) \mapsto X\cup A \in \mathcal{P}(E)$. $\phi$ est-elle injective ? surjective ? Quelle est son image ?
\end{enumerate}

\subsection*{Exercice 7} 

Soit $E$ un ensemble non vide et $A, B \subseteq E$ et $\phi : X \in \mathcal{P}(E) \mapsto (X\cap A, X\cap B) \in \mathcal{P}(A) \times \mathcal{P}(B)$.

\begin{enumerate}
    \item Montrer que $f$ injective $\Leftrightarrow \ A\cup B = E$.
    \item Montrer que $f$ surjective $\Leftrightarrow \ A \cap B = \varnothing$
    \item Trouver une CNS pour que $f$ soit bijective et sous cette condition, exhiber $f^{-1}$
\end{enumerate}

\subsection*{Exercice 8} 

Soit une fonction $f:E\to F$. Montrer que $f$ est bijective ssi $\forall A \subseteq E, f(A^c) = f(A)^c$ 

\subsection*{Exercice 9} 

Montrer que $f: (n,p)\in \N^2 \to 2^n(2p+1)\in \N^* $ est bijective. 

\subsection*{Exercice 10} 

Soit $E$ un ensemble. Soit $f : E \to E$. On suppose que $f\circ f\circ f = f$. Montrer que $f$ est injective ssi elle est surjective. 

\subsection*{Exercice 11} 

Soit $E$ un ensemble non vide. Soit $f : \mathcal{P}(E) \to \mathcal{P}(E)$ non décroissante. Montrer que $f$ admet un point fixe. 

\clearpage

\section{Raisonnement par réccurence et relations binaires}

\subsection{Questions de cours}
\setlength{\parindent}{0cm}

\begin{enumerate}
    \item Montrer que les classes d'équivalence d'une relation d'équivalence d'un ensemble $E$ forment une partition de $E$ 
    \item Montrer que $g\circ f$ surjective $\Rightarrow \ g$ surjective et $g\circ f$ injective $\Rightarrow \ f$ injective 
    \item Montrer que la relation de congruence modulo $n$ est une relation d'équivalence sur $\Z$. 
\end{enumerate}

\subsection{Exercices}

\subsubsection{Réccurences}  
\setlength{\parindent}{0cm}

\subsection*{Exercice 1} 

On définit la suite $(u_n)_{n\in \N}$ par $u_0 = 1 , u_1 = 3$ et $\forall n \in \N \ u_{n+2} = 3u_{n+1} - 2u_n$. Déterminer $u_n$ pour tout $n\in \N$ 

\subsection*{Exercice 2} 

On définit la suite $(u_n)_{n\in \N}$ par $u_0 = 1$ et $\forall n \geqslant 1 \ u_n = \displaystyle\sum_{k=0}^{n-1}{u_k}$. Montrer que $\forall n\in \N^* \ u_n = 2^{n-1}$ 

\subsection*{Exercice 3} 

On définit la suite $(u_n)_{n\in \N}$ par $u_0 = u_1 = 1$ et $\forall n \in \N^* \ u_{n+1} = u_n + \displaystyle\frac{2}{n+1}u_{n-1}$. Montrer que $\forall n \in \N^* \ 1 \leqslant u_n \leqslant n^2$ 

\subsection*{Exercice 4} 

Soit $x \in \R$. Montrer que si $x + \displaystyle\frac{1}{x} \in \Z$ alors $\forall n\in \N \
x^n + \displaystyle\frac{1}{x^n} \in \Z$ 

\subsection*{Exercice 5} 

Montrer que pour tout entier $n\geqslant 3 \ \exists (x_1, \cdots, x_n) \in \N^n$ (avec les $x_i$ 2 à 2 distincts) tel que $\displaystyle\sum_{k=1}^n{\frac{1}{x_k}} = 1$ 

\subsubsection{Relations binaires}

\subsection*{Exercice 1} 

On considère l'ensemble des suites $\R^{\N}$. Montrer que la relation définie sur cet espace par $u \sim v \Leftrightarrow \  u_n - v_n \longrightarrow 0$ est une relation d'équivalence. Trouver la classe d'équivalence de la suite $u$ définie par $\forall n\in \N \ u_n = 3 $ 

\subsection*{Exercice 2} 

Montrer que la relation sur $\Z$ définie par $x\sim y \Leftrightarrow x+y \in 2\Z$ est une relation d'équivalence. Déterminer ses classes d'équivalence. 

\subsection*{Exercice 3} 

Montrer que si $f : E \to F$ est une fonction, $\{f^{-1}(\{y\}), \ y\in f(E)\}$ forme une partition de $E$. 

\clearpage

\section{Fonctions usuelles et notion de groupe}

\subsection*{Questions de cours} 

\begin{enumerate}
    \item Soit $f$ un morphisme de groupe de $G$ vers $G'$. 
Montrer que si $H$ est un sous-groupe de $G$, alors $f(H)$ est un sous-groupe de $G'$. Montrer que si $H'$ est un sous-groupe de $G'$, $f^{-1}(H')$ est un sous-groupe de $G$. Donner les relations coefficients/racines.   
    \item Montrer que le noyau d'un morphisme $f : G \to G'$ est un sous-groupe de $G$. Montrer que $f$ est injectif ssi $\ker{f} = \{e\}$. Formulaire : $\cos(p) + \cos(q)$ et $\arctan'$
    \item Si $f : G \to G'$ et $g : G' \to G''$ sont des morphismes, montrer que $g\circ f$ en est un. Montrer que si $f$ est un morphisme bijectif, $f^{-1}$ est aussi un morphisme. Donner l'énoncé de l'égalité des accroissements finis. 
\end{enumerate}
   
\subsection*{Exercice 1} 

Trouver une relation polynomiale entre les fonctions $x : t \mapsto \cos(2t)$ et $y : t \mapsto \cos(3t)$ 

\subsection*{Exercice 2} 

Montrer que $\ln$ n'est pas le quotient de deux polynômes. 

\subsection*{Exercice 3} 

Montrer que $\displaystyle x\in \R^* \mapsto \arctan{x} + \arctan{\frac{1}{x}}$ est constante sur $\R_+^*$ et sur $\R_-^*$ 

\subsection*{Exercice 4} 

Calculer $\arctan(2) + \arctan(5) + \arctan(8)$ 

\subsection*{Exercice 5} 

Soit $p\in \R_+$. Montrer que $\displaystyle \arctan(p+1) - \arctan(p) = \arctan\bigg(\frac{1}{p^2+p+1}\bigg)$. Quelle est la nature de la suite définie par $\forall n\in \N, S_n = \displaystyle\sum_{p=0}^{n}{\arctan{\frac{1}{p^2+p+1}}}$ ?

\subsection*{Exercice 6} 

Montrer qu'il n'existe pas de bijection continue de $[0, 1]$ dans $\R$ 

\subsection*{Exercice 7} 

Soit $x_1,\dots, x_n$ dans $\R$ tel que $\sum_{i=1}^n{x_i} = \sum_{i=1}^n{x_i^2} = n$. Montrer que pour tout $i \in \llbracket1,n\rrbracket \ x_i=1$. 

\subsection*{Exercice 8}  

Soit $G$ un groupe fini (i.e un groupe dont le cardinal est fini). Soit $f : (G \to \C^*$ un morphisme de groupe. Montrer que $f(G) \subseteq \U = \{z \in \C \ |z| = 1\}$ 

\subsection*{Exercice 9} 

Montrer qu'une fonction continue de $[a, b]$ dans $[a,b]$ admet un point fixe. 

\subsection*{Exercice 10} 

Résoudre les équations suivantes. 

\begin{enumerate}
    \item $\arcsin{x}= \arcsin{4/5} + \arcsin{5/13}$ 

    \item $2\arcsin{x} = \arcsin{2x\sqrt{1-x^2}}$ 

    \item $\arccos{x}= \arcsin{2x}$ 

    \item $\arccos{(1-x)/(1+x)} + \arccos{2\sqrt{x}/(1+x)} = \pi$ 

    \item $\arctan{x}+\arctan{x\sqrt{3}} = 7\pi/12$ 
\end{enumerate}

\clearpage

\section{Corps des nombres complexes}

\subsection*{Exercice 1} 

Soit $n\in \N$. Résoudre dans $\C$ l'équation $(z^2+1)^n = (z-1)^{2n}$ 

\subsection*{Exercice 2} 

Donner la forme exponentielle de
$\displaystyle\frac{1+\cos{\theta} -i\sin{\theta}}{1-\cos{\theta}+i\sin{\theta}}$ et de $\displaystyle\frac{1+e^{i\theta}}{1-e^{i\theta}}$ 

\subsection*{Exercice 3} 

On considère l'équation $(z-1)^n = (z+1)^n$ pour un entier fixé $n\geqslant 2$. 

\begin{enumerate}
    \item Montrer que les solutions sont imaginaires pures. 
    \item Montrer que les solutions sont deux à deux opposées. 
    \item Résoudre l'équation. 
\end{enumerate}



\subsection*{Exercice 4} 

Soit $\alpha \in ]0, \pi/2[$. Résoudre dans $\C$ l'équation : $\displaystyle \left(\frac{1+iz}{1-iz}\right) = \frac{1+i\tan{\alpha}}{1-i\tan{\alpha}}$. 

\subsection*{Exercice 5} 

Déterminer les entiers naturels $n$ pour lesquels $(1+i\sqrt{3})^n$ est un réel positif. 

\subsection*{Exercice 6} 

Posons $Z = \displaystyle\frac{1+z}{1-z}$. Déterminer l'ensemble des points $M$ d'affixe $z$ tel que :
\begin{enumerate}
    \item $|Z| = 1$ 
    \item $|Z| = 2$ 
    \item $Z \in \R$ 
    \item $Z \in i\R$ 
\end{enumerate}


\subsection*{Exercice 7} 

Soit $a \in \U$. On note $z_1, \cdots, z_n$ les racines n-ième de $a$. Montrer que les points d'affixes $(1+z_1)^n, \cdots (1+z_n)^n$ sont alignés. 

\subsection*{Exercice 8} 

Trouver tous les nombres complexes $z$ tels que les points d'affixe $z, z^2$ et $z^4$ sont alignés. 

\subsection*{Exercice 9} 

Calculer $\displaystyle\prod_{k=1}^{n-1}{\sin \left(\frac{k\pi}{n}\right)}$ en considérant le polynôme $P = \displaystyle\prod_{k=1}^{n-1}{(X-e^{\frac{2i k\pi}{n}})}$ 

\subsection*{Exercice 10} 

Soit $(z_i)_{1\leq i \leq n}, (w_i)_{1\leq i \leq n} \in \{z \in \C \ : \ |z| \leq 1\}^n$. Montrer que $\displaystyle \left|\prod_{i=1}^n{z_i} - \prod_{i=1}^n{w_i}\right| \leq \sum_{i=1}^n{|z_i - w_i|}$. 

\clearpage

\section{Équations différentielles et calcul d'intégrales}

\subsection*{Questions de cours}

\begin{enumerate}
   \item Montrer la proposition 14 cas où $r_1 = r_2 = r$ 
    \item Montrer que le noyau d'une application linéaire $f:E\to F$ est un sous espace vectoriel de $E$. En déduire que l'ensemble des solutions d'une EDH est un sev de $C^1(I, \K)$ 
    \item  Proposition 17 cas où $b=0, a \neq 0$ 
\end{enumerate}


\subsection*{Exercice 1 (Lemme de Riemann-Lebesgue)} 

Soient $a, b$ deux réels. Soit $f : [a, b] \to \R$ de classe $C^1$. Montrer que $\displaystyle\int_a^b{f(t)\cos{(nt)}dt} \longrightarrow 0 $ quand $n \rightarrow +\infty$  

\subsection*{Exercice 2} 

Soient $(p,q)\in \N^2$. Calculer $I(p,q) = \displaystyle\int_0^1{x^p(1-x)^qdx}$. (\textbf{solution} : $p!q!/(p+q+1)!$) 

\subsection*{Exercice 3 (Lemme de Gronwall)} 

Soit $g \geqslant 0$ et $f$ deux fonctions continues et $K\in \R_+$ telles que $\forall t\geqslant t_0\ f(t) \leqslant K + \displaystyle\int_{t_0}^t{f(s)g(s)ds}$. Montrer que $\forall t\geqslant t_0 \ f(t) \leqslant K\exp{\left(\displaystyle\int_{t_0}^t{g(s)ds} \right)}$. 

En déduire une majoration d'une fonction $y$ vérifiant $|y'| \leqslant \alpha + \beta |y|$. 

(\textbf{Indication :} Poser $F : t \mapsto K + \displaystyle\int_{t_0}^t{f(s)g(s)ds}$, multiplier par $g$ dans les inégalités. On tombe sur $y' - a'y \leqslant 0$ on multiplie donc par $e^{-\int{a}}$) 

\subsection*{Exercice 4} 

On considère le problème de Cauchy suivant : $(1-x)y' - xy = 0 \ \forall x\in ]-1, 1[$ et $y(0) = 1$. 
\begin{enumerate}
    \item Résoudre le problème. On note $f$ son unique solution. 

    \item Jusifier que $f \in C^{\infty}(]-1, 1[, \R)$. 

    \item Pour tout $x \in ]-1, 1[$ expliciter $(1-x)f(x)$ puis exprimer pour tout entier $n\in \N \ \ (1-x)f^{(n+1)}(x) - (n+1)f^{(n)}(x)$ en fonction de $n$ et de $x$. 
\end{enumerate}

\subsection*{Solution :} 

\begin{enumerate}
    \item  $f(x) = \displaystyle\frac{e^{-x}}{1-x}$ (3) En utilisant la formule de Leibniz : $(1-x)f^{(n+1)}(x) - (n+1)f^{(n)}(x) = (-1)^{n+1}e^{-x}$ 
\end{enumerate}

\subsection*{Exercice 5} 

Résoudre les équations différentielles suivantes :

\begin{enumerate}
   \item$y'' - 4y + 4y = x$ (solution : $(\alpha x + \beta)e^{2x} + (x+1)/4$)
\item$y'' + 2y' + 4y = xe^x$ (solution : $\alpha \sin(\sqrt{3}x) + \beta \cos(\sqrt{3}x) + e^x(x/7  - 4/49)$)
\item$2y'' - 3y' + y = e^x$ (solution : $\alpha e^{x/2} + \beta e^{x} + xe^x$) 
\item$y'' + y = x^2$ (solution : $\alpha \cos{x} + \beta \sin{x} + x^2 - 2$)
\item$y'' - 2y' - 3y = 0$ 
\item$y'' - 2y' + y = 0$  

\end{enumerate}

\subsection*{Exercice 6} 

Soit $a, b$ deux fonctions continues respectivement impaire et paire. Montrer que $(E) \ y' + ay = b$ possède une unique solution impaire.  

\clearpage

\section{Corps des nombres réels} 

\subsection*{Question de cours} 

\begin{enumerate}
    \item Montrer que $f$ est injectif si et seulement si $\ker{f} = \{e\}$. 
    \item Caractérisation de la borne supérieure. 
    \item Les intervalles de $\overline{\R}$ sont les parties convexes de $\overline{\R}$. 
    \item Soit $f$ un morphisme bijectif. Montrer que $f^{-1}$ est aussi un morphisme. 
    \item L'ensemble des unités d'un anneau forme un groupe. 
    \item $x$ n'est pas régulier à gauche ssi $x$ n'est pas un diviseur de zéro à gauche. 
\end{enumerate}

Donner les exos 1 et 3 sur les groupes. 

\subsection*{Exercice 1} 

On pose $A = \left\{1/n + (-1)^n \forall n \in \N^* \right\}$. Déterminer $\sup{A} $ et $ \inf{A}$


\subsection*{Exercice 2} 

Soit $f : \R \to \R$ une fonction périodique. Montrer que l'ensemble des périodes de $f$ est un sous-groupe de $(\R, +)$. Montrer que si $f$ est continue et que $1$ et $\sqrt{2}$ sont deux périodes de $f$ alors $f$ est constante. 

\subsection*{Exercice 3} 

Montrer que $\{x^3 \ | \ \forall x \in \Q \}$ est dense dans $\R$. 

\subsection*{Exercice 4} 

Soit $f : [0, 1] \to [0, 1]$ une application croissante. On pose $E = \{ x\in [0, 1] \ | \ f(x) \geqslant x \}$. Montrer que $E$ admet une borne supérieure $b$ et que $f(b) = b$. Si $f$ est décroissante, admet-elle un point fixe ? 

\subsection*{Exercice 5} 

Soient $(a, b) \in \R^2$. Déterminer les bornes supérieures et inférieures (dans le cas où celle-ci existe) des ensembles suivant : 

\begin{enumerate}
    \item $\{ a + bn, n\in \N\}$ 
    \item $\{ a + b/n, n\in \N^*\}$ 
    \item $\{ a + (-1)^n b/n, n\in \N^*\}$ 
    \item $\{ n/(mn+1), (m,n)\in (\N^*)^2\}$ 
    \item $\{ n/(mn+1), (m,n)\in \N^2\}$    
\end{enumerate}


\subsection*{Exercice 6} 

Soient deux réels $a, b > 1$. On pose $E = \{ \lfloor an \rfloor , \ n \in \N^* \}$ et $F = \{ \lfloor bn \rfloor , \ n \in \N^* \} $. Montrer l'équivalence entre : 

\begin{enumerate}
    \item $E$ et $F$ forment une partition de $\N^*$. 
    \item $\displaystyle\frac{1}{a} + \frac{1}{b} = 1$ et $a$ et $b$ sont irrationnels. 
\end{enumerate}

\subsection*{Exercice 7} 

Soient $a_1, \dots, a_n \in \R_+$. Montrer que $\displaystyle \prod_{i=1}^n{(1+ a_i)} \geqslant 2^n$ 

\subsection*{Exercice 8} 

Soit $F : A\times B \to \overline{\R}$. Montrer que $\underset{x\in A }{\sup} \ \underset{y \in B}{\inf} F(x, y) \leq \underset{y \in B}{\inf} \ \underset{x \in A}{\sup} F(x, y) $ 

On dit que $(x^*, y^*) \in A\times B$ est un point selle si $\forall (x, y) \in A\times B \ F(x,y^*) \leq F(x^*, y^*) \leq F(x^*, y)$. Montrer que s'il existe un point-selle alors il y a égalité dans l'inégalité montrée précédemment.  

\clearpage

\section{Suites numériques} 

\subsection*{Question de cours} 

\begin{enumerate}
    \item Montrer le Lemme de Cesaro. 
    \item Montrer le Théorème des suites adjacentes. 
    \item Montrer le Théorème des gendarmes. 
\end{enumerate}

\subsection*{Exercice 1 (Suites de Cauchy)} 

\begin{enumerate}
    \item Soit $(u_n)_{n}$ une suite de Cauchy. Montrer qu'elle est bornée. 
    \item On suppose que $u$ admet une valeur d'adhérence. Montrer que $u$ converge. 
\end{enumerate}    
\subsection*{Exercice 2 (Lemme sous-additif)} 

Soit $(u_n)$ une suite telle que $\forall n, m \ u_{n+m} \leqslant u_n + u_m$ 

Montrer que $ \displaystyle \frac{u_n}{n}$ converge vers $\displaystyle \inf \left\{\frac{u_k}{k}, k \in \N^*\right\}$. (\textbf{indication :} écrire la définition de la borne supérieure et poser une division euclidienne) 

\subsection*{Exercice 3} 

Soit $(a, b) \in \R_+^*$ tel que $0 < a < b$. On pose $x_0 = a$ et $y_0 = b$. $x_{n+1} = (x_n+y_n)/2, y_{n+1} = 2x_ny_n/(x_n+y_n)$. Déterminer les limites des suites $(x_n)_n$ et $(y_n)_n$. 

\subsection*{Exercice 4} 

Déterminer toutes les fonctions $f : \R_+^* \to \R_+^*$ telles que $\forall x > 0 \ f(f(x)) = 6x - f(x)$. (\textbf{indication} considérer $u_n = f^n(x)$ ) Écrire la solution. 

\subsection*{Exercice 5} 

Déterminer le nombre $a_n$ de façons de pavés avec des dominos un damier de dimension $2\times n$. 

\subsection*{Exercice 6 (Échauffement)} 

Donner l'expression du terme général des suites récurrentes suivantes : 

1. $u_{n+2} = 3u_{n+1} - 2u_n, u_0 = 3, u_1 = 5$ 
2. $u_{n+2} = 4u_{n+1} - 4u_n, u_0 = 1, u_1 = 0$ 
3. $u_{n+2} = u_{n+1} - u_n, u_0 = 1, u_1 = 2$ 

\subsection*{Exercice 7} 

Soit $(a_i)_{i\in \N}$ une suite vérifiant $\forall m \geqslant n \ a_{n+m} + a_{m-n} = \displaystyle\frac{1}{2}(a_{2m} + a_{2n})$ et telle que $a_1 = 1$. Que vaut $a_{2023}$ ? 

\subsection*{Exercice 8} 

Soit $K\in \Z$. On définit la suite $(a_i)_{i\in \N}$ par $a_0 = 0, a_1 = K$ et $\forall n\in \N \ a_{n+2} = K^2a_{n+1} - a_n$. Prouver que pour chaque entier $n$, $1 + a_na_{n+1}$ divise $a_n^2 + a_{n+1}^2$. 

\subsection*{Solution} 

Notons $\beta \neq \alpha $ les racines du polynôme caractéristique  $X^2 - K^2X + 1$. On remarque que l'on a $\alpha + \beta = K^2$ et $\alpha\beta = 1$ ce qui sera très utile par la suite. 

On a alors $a_n = A(\alpha^n - \beta^n)$ ce qui donne : 

- $a_n^2 + a_{n+1}^2 = A^2(\alpha^n - \beta^n)^2 + A^2(\alpha^{n+1} - \beta^{n+1})^2  = A^2(\alpha^{2n} + \alpha^{2n+2} + \beta^{2n} + \beta^{2n+2} - 4)
= A^2(\alpha^{2n+1}K^2 + \beta^{2n+1}K^2 - 4) = A^2K^2(\alpha^{2n+1} + \beta^{2n+1} - 4/K^2)  
= K^2(\alpha^{2n+1}A^2 + \beta^{2n+1}A^2 - 4A^2/K^2)$ 

- $a_na_{n+1} + 1 = A^2(\alpha^n - \beta^n)(\alpha^{n+1} - \beta^{n+1}) + 1 = A^2(\alpha^{2n+1} + \beta^{2n+1} - K^2) + 1 = A^2(\alpha^{2n+1} + \beta^{2n+1} - K^2 +1/A^2) 
=  (\alpha^{2n+1} + \beta^{2n+1} - A^2K^2 +1) $  

On a $A^2K^2 - 1 = 4A^2/K^2$ et donc $a_n^2 + a_{n+1}^2 = K^2(a_na_{n+1} + 1) $ et $K^2 \in \mathbb{N}$. 

Donc, pour tout $n\in \mathbb{N}  \ \ a_na_{n+1} + 1$ divise $a_n^2 + a_{n+1}^2$. 


\subsection*{Exercice 9} 

Soit $a \in ]0, 1[$ un réel. Soit $(u_n)_{n\in \N}$ une suite définie par $0 < u_0 < u_1$ et $u_{n+1} = u_n + a^nu_{n-1}$. Montrer que $(u_n)_{n\in \N}$ converge.  

\subsection*{Exercice 10} 

Soit $(u_n)_n$ une suite telle que $\forall n \in \N^* \ u_{n} \leqslant \displaystyle\frac{1}{2}(u_{n-1} + u_{n+1})$.  

(a) Montrer que la suite définie par $\forall n \in \N \ v_n = u_{n+1} - u_n$ est croissante. 

(b) Montrer que si $(u_n)_n$ est bornée alors $v_n \to 0$. (\textbf{indication :} Montrer que $(v_n)_{n\in \N}$ est bornée, noter $\ell$ sa limite et supposer par l'absurde que $\ell \neq 0$) 

\textbf{Remarque} : la propriété de cette suite fait penser à la propriété de convexité d'une fonction réelle. On peut donc faire un dessin de $u_n$ pour se donner une intuition de ce qu'il se passe. $(v_n)$  correspond à la dérivée discrète de $(u_n)$. 

\subsection*{Exercice 11 (Limites supérieure et inférieure)} 

Soit $(u_n)_{n\in\N} \in \R^{\N}$ une suite. On pose $v_n = \inf \{ u_k \ | \  k \geqslant n \}$ et $w_n = \sup\{ u_k \ | \  k\geqslant n \}$. Montrer que $(u_n)$ converge si et seulement si $(v_n)$ et $(w_n)$ convergent et ont même limite et que dans ce cas $\lim{u_n}=\lim{v_n} = \lim{w_n}$.  

\clearpage

\section{Analyse : continuité, dérivation et fonctions convexes} 

\subsection*{Questions de cours} 
\begin{enumerate} 

\item Démontrer la caractérisation séquentielle des limites et de la continuité. 

\item Démontrer le théorème de Heine. 

\item Montrer qu'une fonction lipschitzienne est uniformément continue. 

\item Montrer le théorème de Rolle. 

\item Montrer l'égalité de Taylor Lagrange. 

\item Montrer le théorème de la limite de la dérivée. 
\end{enumerate}

\subsection*{Exercices}

Poser des exos de convexité, des inégalités etc... 

\subsection*{Exercice 1} 

Montrer qu'une fonction convexe majorée sur $\R$ est constante. 

Trouver un exemple de fonction convexe majorée sur $\R_+$ non constante. 

\subsection*{Exercice 2} 

Soit $n$ un entier. On pose $f(x) = \begin{cases} (1-x^2)^n & \text{si} \cr |x|\leqslant 1  0 & \text{sinon} \end{cases}$. Déterminer la classe de $f$. 

\subsection*{Exercice 3} 

Soit $P$ une fonction polynomiale de degré $n$. Montrer que l'équation $P(x) = e^x$ admet au plus $n+1$ solutions. 

\subsection*{Exercice 4} 

Soit $f  : \R \to \R$ dérivable telle que $f^2 + (1+f')^2 \leqslant 1$. Montrer que $f$ est nulle. 

\textbf{indication :} $f$ est bornée et croissante. Si $f(x) \geq \varepsilon$ alors $\forall y \geq x  f'(y) \geq 1 - \sqrt{1 - \varepsilon^2}$ donc $f$ diverge. 


\subsection*{Exercice 5} 

Soit $P$ un polynôme impair et $f \in \mathcal{C}^{\infty}(\R, \R)$ telle que $\forall n \in \N \ |f^{(n)}|\leq |P|$. Montrer que $f$ est nulle. 

\subsection*{Exercice 6} 

Montrer que si $f$ est de classe $\mathcal{C}^2(\R, \R)$ et que $f$ et $f''$ sont bornées par $M_0$ et $M_2$ respectivement alors $f'$ est bornée par $2\sqrt{M_0M_2}$. 

\subsection*{Exercice 7} 

Soient $f, g$ dérivables et convexes sur $[0, 1]$ telles que $\max(f, g)$ soit positive sur $[0, 1]$. Montrer qu'il existe $\alpha \in [0, 1]$ tel que $\alpha f + (1-\alpha)g $ est positive sur $[0, 1]$. 

\subsection*{Exercice 8} 

\begin{enumerate}    
\item Trouver les fonctions de $\mathcal{C}^0(\R, \R)$ telle que $f(2x) = -f(x)$. 

\item Trouver les fonctions de $\mathcal{C}^0(\R, \R)$ telle que $f(2x) = 2f(x)$. 

\item Trouver les fonctions de $\mathcal{C}^0(\R, \R)$ telle que $f(2x) = f(x)$. 
\end{enumerate}
\subsection*{Exercice 9} 

Soit $f$ une fonction continue sur $\R$ telle que $f(x)^2 = 1$. Montrer que $f=1$ ou $f=-1$. 

\subsection*{Exercice 10} 

Soit $f$ une fonction continue sur $\R$. Montrer la fonction $M : x \mapsto \sup\left\{f(t) : t\in [0, x]\right\}$ est une fonction croissante et continue. 

\subsection*{Exercice 11} 

Soit $f, g : [0, 1] \to [0, 1]$ deux fonctions continues telles que $f\circ g = g\circ f$. Montrer qu'il existe $c\in [0, 1]$ tel que $f(c) = g(c)$. 

\subsection*{Solution} 

Prendre un point fixe $s$ de $f$ et poser $u_0 = s, u_{n+1} = g(u_n)$ 

\subsection*{Exercice 12} 

Trouver toutes les fonctions continues $f : \R \to \R$ vérifiant $f(2x) - f(x) = x$. 

\textbf{Indication} 

Exprimer $f(x) - f(x/2^n)$  

\subsection*{Exercice 13} 

Montrer qu'une fonction périodique et continue sur $\R$ est uniformément continue. 

\subsection*{Exercice 14} 

Soit $f : [0, 1] \to \R_+^*$ de classe $\mathcal{C}^1$ telle que $\forall x \in [0, 1], f'(x) \geq f(x)^3$. Montrer que $f(0) \leq 1/\sqrt{2}$. 

\subsection*{Exercice 15} 

Soit $f : \R_+ \to \R$ convexe. Montrer que $t \mapsto f(t)/t$ tend vers une limite finie $\ell$ ou vers $+\infty$ quand $t$ tend vers $+\infty$. Montrer que si la limite est finie $\ell \in \R$ alors $t \mapsto (f(t) - \ell t)$ tend vers une limite finie ou vers $-\infty$. 

\subsection*{Exercice 16} 

Soit $(x_j)_{1\leq j \leq n} \in (\R_+^*)^n$. Montrer que $\displaystyle \frac{x_1}{x_2} + \cdots + \frac{x_n}{x_1} \geq n$ et étudier le cas d'égalité. 

\clearpage

\section{Intégration}

\subsection*{Questions de cours} 

\begin{enumerate}
\item Démontrer l'inégalité de Cauchy-Schwarz intégrale. 

\item Démontrer l'égalité de Taylor-Lagrange avec reste intégral. 

\item Démontrer le théorème des sommes de Riemann. 
\end{enumerate}

\subsection*{Exercice 1} 

Soit $f : [0, 1] \to \R_+^*$ continue. 

1. Etablir l'existence d'une subdivision $\sigma = (a_{n,i})_i$ de $[0, 1]$ telle que $
\forall i \ \displaystyle\int_{a_{n,i}}^{a_{n,i+1}}{f(t)dt} = \frac{1}{n}\int_0^1{f}$ 

2. Étudier le comportement de $\displaystyle\frac{a_{n,0}+\cdots + a_{n,n}}{n+1}$ (indication : utiliser le théorème des sommes de Riemann)

\subsection*{Exercice 2} 

Soit $f : [0, 1] \to \R$ continue telle que la suite $(u_n)_n$ définie par $u_n = \displaystyle\int_0^1{f^n}$ prend un nombre fini de valeurs. Montrer que $f$ est constante. 

\subsection*{Exercice 3} 

Soit $a > 0$ et $f : [0, a] \to \R$ de classe $C^1$ telle que $f(0) = 0$. 

\begin{enumerate}
\item Montrer que $\displaystyle\int_0^a{|ff'|} \leqslant \frac{a}{2} \int_0^a{f'^2}$ 
\item Étudier les cas d'égalité. 
\end{enumerate}
\subsection*{Exercice 4} 

Trouver un équivalent quand $n \to +\infty$ de $u_n = \displaystyle\sum_{k=0}^n{\frac{1}{k^2 + (n-k)^2}}$ 

\subsection*{Exercice 5} 

Soit $f : [0, 1] \to \R$ continue telle que $\displaystyle\int_0^1{f^2} = \int_0^1{f^3} = \int_0^1{f^4} $. Montrer que $f = 0$ ou $f = 1$. 

\subsection*{Exercice 6} 

Montrer le lemme de Gronwall. 

\subsection*{Exercice 7} 

Soit $f : [0, 1] \to \R$ de classe $C^2$. Montrer que $\displaystyle \left(\int_0^1{f'^2}\right)^2 \leqslant \left(\int_0^1{f^2}\right)\left(\int_0^1{f''^2}\right) $ 

\subsection*{Exercice 8} 

Calculer $\displaystyle\sum_{n=1}^{+\infty}{\frac{(-1)^n}{n}}$. 

\subsection*{Exercice 9} 

Soit $f : [0, 1] \to \R$ telle que $f(0) = 0$ et $f(1) = 1$ de classe $C^1$. Montrer que $\displaystyle \int_0^1{|f' - f|} \geq \frac{1}{e}$. 

\subsection*{Exercice 10} 

Soit $f : [0, 1] \to \R$ telle que $f(0) = 0$ de classe $C^1$. Montrer que $\displaystyle \int_0^1{f^2} \leq \frac{1}{2}\int_0^1{f'^2} $. Étudier le cas d'égalité. 

\subsection*{Exercice 11} 

On pose $$\forall n\in \N : I_n = \int_0^1{\frac{x^n}{x+10}dx}$$.

\begin{enumerate}
    \item Montrer que $I_n \longrightarrow 0$. 
    \item Trouver une relation de récurrence entre $I_{n+1}$ et $I_n$. 
    \item Trouver un équivalent de $I_n$. 
\end{enumerate}

À rajouter : Exercices 30, 31, 32, 34 

\clearpage

\section{Développements limités} 

\subsection*{Exercice 1 (Oral des Mines 2021)} 

On définit la suite $(u_n)_{n\in \N}$ par récurrence. $u_0 = x \in \R^+$ et $u_{n+1} = \displaystyle\sqrt{1 + \Bigg(\sum_{k=0}^n{u_k}\Bigg)^2}$. 

\begin{enumerate}{}
    \item Montrer que pour tout $n\in \N$ il existe un unique $\theta_n \in ]0, \displaystyle\frac{\pi}{2}]$ tel que $u_{n+1} = 1/\sin{\theta_n}$. Montrer que $\displaystyle \frac{1}{\tan{\theta_{n+1}}} = \frac{1}{\tan{\theta_n}} + \frac{1}{\sin{\theta_n}}$ 
    \item Déterminer $\theta_n$ puis donner un équivalent de $u_n$. 
\end{enumerate}

\textbf{indication} : Pour la (1) : utiliser la formule de récurrence en calculant $\tan(\theta_n)$ en fonction des $u_k$. Pour la (2) : transformer le $\tan$ en $\sin/\cos$ puis utiliser une formule de trigonométrie.  

\clearpage

\section{Dénombrements et Probabilités sur un univers fini} 

\subsection{Dénombrement}

exercice serrage de mains, nombre de catalans sans séries entières, pavage d'un échiquier privé des cases gauche/haut et droit en bas, avec des dominos  

\subsection*{Exercice 1 (Anagrammes)} 

1. Compter le nombre d'anagrammes du mot PERMUTATION telles que les voyelles sont dans l'ordre alphabétique.  

2. Compter le nombre d'anagrammes du mot ANAGRAMME. 

\subsection*{Exercice 2}  

Soit $n\geqslant 2$ un entier. $n$ personnes se serrent la main (certaines personnes peuvent ne pas se serrer la main). Montrer que deux personnes du groupe ont serré le même nombre de main.    

\subsection*{Exercice 3 (nombre d'involutions)}  

Déterminer une relation de récurrence pour le nombre d'involutions de $\mathfrak{S}_n$. 

\subsection*{Exercice 4} 

Montrer la formule de Vandermonde. 

nombres de dérangements, nombres de Stirling, nombres de Catalans 

\subsection*{Exercice 5 (nombres de Stirling)} 

Compter le nombre de surjections de $\n$ dans $\k$. En déduire le nombre de Stirling $\begin{Bmatrix}
  n 
  k 
\end{Bmatrix}$ (le nombre de partitions de $\n$ en $k$ éléments). 

\textbf{Solution} 

Une première méthode est d'utiliser l'union suivante : $S_{n, k} = \displaystyle \bigcup_{p=1}^{n-k+1} \{ f \in S_{n, k} | \#f^{-1}\{k\} = p \}$ 

Une deuxième méthode est de compter le nombre de non-surjections : on choisit un élément $p\geqslant 1$ éléments parmi $k$ qui ne vont pas avoir d'antécédent. Il nous reste à choisir une surjection de $\n$ dans $\llbracket 1, k-p \rrbracket$. 

\subsection*{Exercice 6} 

Combien y a-t-il de parties de $\left\{1, \dots, n \right\}$ à $k$ éléments ne contenant pas d'entiers consécutifs ? 

\subsection{Probabilités}

\subsection*{Exercice 7 (Entropie)} 

\clearpage

\section{Structures algébriques} 

\subsection*{Exercice 1} 

Soit $G$ un groupe. Soit $H$ et $K$ deux sous-groupes de G tels que $H\cup K$ est un groupe. Montrer que $H \subseteq K$ ou que $K \subseteq H$. 

\subsection*{Exercice 2} 

Soit $G$ un groupe non commutatif. Montrer que la probabilité que deux éléments $x, y$ dans $G$ commutent est inférieur à $5/8$. ($\#\{(x, y)\in G^2 \ | \ xy = yx\})/\#G^2 \leqslant 5/8$). 

\subsection*{Exercice 3} 

Soit $G$ un groupe tel que $\forall x \in G \ x^2 = e$. Montrer que $G$ est abélien. 

\subsection*{Exercice 4} 

On note $\mathbb{H} = \left\{ M_{a,b} = 
\begin{pmatrix} 
    a & b\\
    -\overline{b} & \overline{a} 
    
\end{pmatrix}  |  (a,b)\in \C^2 \right\}$. Montrer que $\mathbb{H}$ est un sous-groupe non abélien de $GL_2(\C)$. 

\clearpage

\section{Arithmétique dans un anneau intègre} 

\subsection*{Questions de cours} 

\begin{enumerate}
    \item Démontrer l'identité de Bézout dans un anneau principal. 
    \item Énoncer l'algorithme d'Euclide. Démontrer sa correction et sa terminaison. 
    \item Démontrer que si $K$ est un corps alors $K[X]$ est un anneau principal. 
\end{enumerate}
\subsection*{Exercice 1} 

Existe-t-il une fonction $f : \N \to \N$ telle que $\forall (x,y)\in \N^2 \ f(x)^{f(y)} = y^x$ ? (Indication : choisir un bon couple $(x, y) \in \N^2$) 

\subsection*{Exercice 2} 

Montrer que si $p$ et $p^2+8$ sont premiers alors $p^3 + 8p + 2$ l'est également. (\textbf{indication :}  Tester pour de petits nombres  premiers $p$) 

\subsection*{Exercice 3} 

Montrer que si l'on divise un nombre premier par $30$ alors le reste de la division est soit $1$ soit un nombre premier. 

\textbf{Solution} 

Soit $p$ un nombre premier.
-Si $p < 30$ alors $ p\equiv p [30] $ et p est un nombre premier.
-Si $p > 30$ alors $p\equiv r[30]$ avec $r<30$
Supposons par l'absurde que $r$ ne soit pas premier et différent de $1$. On vérifie facilement que $2, 3$ ou $5$ divise $r$. Supposons par exemple que $5 | r$ (les autres cas se traitent de la même façon) alors, il existe $k, l > 1$ tels que :
$p = 5l + 30\cdot k = 5(l + 6\cdot k)$ donc $p $ n'est pas premier ce qui est absurde. Donc $r$ est premier ou égal à $1$.
Si $p$ est un nombre premier, alors le reste de sa division par 30 est soit 1 soit un nombre premier.
Ce résultat ne se généralise pas avec $60$ et $90$ en effet $109$ et $139$ sont premiers et pourtant :
$109 = 60 + 49 $ et $139 = 90 + 49$
Or $49$ n'est pas premier et clairement différent de $1$. 

\subsection*{Exercice 4} 

Soit $(n, m) \in \N^2$. On suppose que $24$ divise $1+nm$. Montrer que $24$ divise $n+m$. 

\textbf{Solution} 

Montrons que $n+m$ est divisible par $24$. on a $24 = 3\cdot 8$ et $3$ et $8$ sont premiers entre eux. Ils suffit donc de montrer que $3$ divise $n+m$ et que $8$ divise $n+m$.
Comme $nm = -1 \mod 3 $ on a que $n, m \neq 0 \mod 3$. De plus, si $x \neq 0 \mod 3$ alors $x^2 = 1 \mod 3$. Ainsi $(n+m)^2 = n^2 + m^2 + 2nm = 1 + 1 - 2 \mod 3 = 0 \mod 3$. Donc $3$ divise $(n+m)^2$ et donc $3$ divise $n+m$.
En écrivant sur une feuille la table de multiplication de $\mathbb{Z}/8\mathbb{Z}$ on s'aperçoit que si $nm = 7 \mod 8$ alors $\{\overline{n}, \overline{m} \} = \{ 3, 5\}$ et donc $n+m = 0 \mod 8$. 

On a donc que $3$ et $8$ divisent $n+m$ donc $24$ divise $n+m$. 

\subsection*{Exercice 5 (Équation de Pell)} 

On considère l'équation diophantienne $x^2 - Ky^2 = 1$ où $K$ est un entier qui n'est pas un carré parfait. 
Montrer que s'il existe une solution $(x_0, y_0) \in (\N^*)^2$ alors l'équation admet une infinité de solutions entières. 

Question intermédiaire : montrer que $\Z[\sqrt{K}] = \{ a + \sqrt{K}b, (a, b) \in \Z^2 \}$ est un anneau. 

\subsection*{Exercice 6 (Équation diophantienne de degré 2)} 

Résoudre dans $\mathbb{Z}^2$ l'équation $x^4 - y^2 = 17$. 

\subsection*{Exercice 7 (Critère de divisibilité)} 

Montrer les critères de divisibilité par $7$ et $17$ : 

1. $17$ divise $\overline{a_na_{n-1}\dots a_0} \Longleftrightarrow 17$ divise $\overline{a_na_{n-1}\dots a_1} - 5a_0$ 


2. $7$ divise $\overline{a_na_{n-1}\dots a_0} \Longleftrightarrow 7$ divise $\overline{a_na_{n-1}\dots a_1} + 5a_0$ 

\textbf{Solution pour 17} 

Montrons que $5\cdot 10^n = -10^{n-1} \mod 17 \ \forall n \in \mathbb{N}^*$. On a $50 = 3\cdot17 - 1 = -1 \mod 17$, il suffit donc de multiplier par $10^{n-1}$. 

On a donc les équivalences suivantes (qui sont valables car $17$ est premier donc $\mathbb{Z}/17\mathbb{Z}$ est un corps et en utilisant la propriété montrée précédemment) : 

$\displaystyle \sum_{k=0}^n{a_i 10^i} = 0 \mod 17 \Longleftrightarrow -5\cdot\sum_{k=0}^n{a_i 10^i} = 0 \mod 17 \Longleftrightarrow \sum_{k=1}^n{a_i 10^{i-1}}- 5a_0 = 0 \mod 17$ 

Cela montre bien le critère de divisibilité énoncé. 

\subsection*{Exercice 8} 

Trouver toutes les paires d'entiers naturels $(x, y)$ tels que $x \leqslant y$ et $\mathrm{pgcd}(x, y) = 5!$ et $\mathrm{ppcm}(x, y) = 50!$. 

\subsection*{Exercice 9} 

Quel est le plus grand diviseur commun aux nombres $p^8 - 1$ avec $p$ premier strictement supérieur à 5 ?  

\clearpage

\section{Polynômes} 

\subsection*{Questions de cours} 

\begin{enumerate}
    \item Démontrer la proposition 27 sur la comparaison de divisibilité. 
    \item Démontrer la formule de Taylor. 
    \item Démontrer que si $K$ est un corps alors $K[X]$ est un anneau principal. 
\end{enumerate}

\subsection*{Exercice 1} 

Trouver les polynômes $P$ de $\R[X]$ tels que $(X-16)P(2X) = 16(X-1)P(X)$ 

\subsection*{Exercice 2} 

Trouver l'ensemble des solutions réelles de l'équation $(x+1995)(x+1997)(x+1999)(x+2001) + 16 = 0$ 

\textbf{Solution} 

Soit $x\in \mathbb{R}$ vérifiant l'équation de l'énoncé.

On pose $y = x + 1998$ ce qui nous donne $(y-3)(y-1)(y+1)(y+3) + 16 = 0 = (y^2 - 9)(y^2 - 1) + 16 = y^4 - 10y ^2 + 25 = Y^2 - 10Y + 25$ en posant $Y = y^2$ 

$0 = Y^2 - 10Y + 25 = (Y - 5)^2 - 25 + 25 = (Y-5)^2$ et donc nécessairement $y^2 = 5$ ce qui nous donne $y \in \{ -\sqrt{5}, +\sqrt{5} \}$ et donc $x \in \{ -\sqrt{5} - 1998, \sqrt{5}+1998 \}$ et $ -\sqrt{5} - 1998, \sqrt{5}-1998$ sont bien des solutions de l'équation. 

L'ensemble des solutions réelles de l'équation est donc $\{ -\sqrt{5} - 1998, \sqrt{5}-1998 \}$ 

\subsection*{Exercice 3} 

Soit $P \in \R[X]$ de degré $3$ tel que $P(19) = 3, P'(19) = 0, P''(19) = 8, P^{(3)}(19) = 18$. Que vaut $P(16)$ ? (indication : utiliser la formule de Taylor) 

\subsection*{Exercice 4} 

Soit $a < b$ deux réels. On définit la longueur des intervalles $[a, b], ]a,b[, ]a, b], [a, b[$ par $\ell([a, b]) = b-a$. 

Montrer que l'ensemble des solutions de l'inéquation $\displaystyle\sum_{k=1}^{70}{\frac{k}{x-k}} \geqslant \frac{5}{4}$ est une union disjointe d'intervalles dont la somme des longueurs vaut $1988$. 

\subsection*{Solution} 

écrire la solution... 

\subsection*{Exercice 5} 

Montrer qu'un corps fini n'est jamais algébriquement clos. 

\subsection*{Solution} 

Soit $P$ admettant tous les éléments de $K$ comme racines. Alors $P+1$ n'a pas de racine. 

\subsection*{Exercice 6} 

Trouver tous les polynômes $P$ tels que $P(\U) \subseteq \U$. 

\subsection*{Exercice 7} 

Trouver tous les polynômes $P$ tel que $P$  divise $P^2$. 

\subsection*{Exercice 8} 

Combien y-a-t-il de polynômes de $\R_{n-1}[X]$ tels que $X^n -1  | \ P^2 - X$ ? 

\subsection*{Exercice 9} 

Soit $P \in \R_n[X]$ tel que $\forall k \in \n \ P(k) = k/(k+1)$. Combien vaut $P(n+1)$ ? 

\subsection*{Exos Biolley} 

\clearpage

\section{Algèbre linéaire} 

\subsection*{Questions de cours} 

\begin{enumerate}
    \item Montrer la caractérisation de la somme directe. 
    \item Montrer la proposition 8. 
    \item Montrer la proposition 12. 
    \item Montrer le théorème 2. 
    \item Montrer la proposition 32. 
    \item Montrer la proposition 35. 
\end{enumerate}
\subsection*{Exercice 1} 

Soit $E, F$ deux $K$ espaces vectoriels, $f \in \mathcal{L}(E, F)$ et $V, W$ deux s.e.v de $E$. Montrer que : $$f(V) \subseteq f(W) \Longleftrightarrow V + \ker(f) \subseteq W + \ker(f)$$  

\subsection*{Exercice 2} 

Lesquels de ces espaces sont des espaces vectoriels ? (Détailler le corps, les lois de groupe et externe) : 

$(i) \left\{ (x,y,z)\in \R^3  | \ x - 2y + 5z = 2 \right\}$ 

$(ii) \left\{ f \in \mathcal{C}^0([a, b], \R) \ | \ \displaystyle\int_a^b{f} = 0 \right\}  $  

$(iii) \  \R_n[X] $ 

\subsection*{Exercice 3} 

Soit $E$ un espace vectoriel de dimension finie $n$. Montrer que pour $f \in \mathcal{L}(E)$ on a l'équivalence : 

$$\ker{f} = \text{Im}{f} \Longleftrightarrow f\circ f = 0 \text{ et } \exists h \in \mathcal{L}(E), f\circ h + h\circ f = \text{Id} $$ 

\subsection*{Exercice} : sur les unions de sev stricts 



\subsection*{exos Biolley} 

\clearpage

\section{Matrices} 

\subsection*{Exercice 1} 

Soit $V$ un $\mathbb{K}$ espace vectoriel de dimension finie $d\in \N^*$. Soit $G$ un sous-groupe fini de $GL(V)$. On pose l'ensemble : $V^G = \{x\in V \ | \ \forall g \in G \ g(x) = x\}$. Montrer que $\displaystyle\frac{1}{|G|}\sum_{g\in G}{\text{Tr}(g)} = \dim{V^G}$. 

\subsection*{Exercice 2} 

Soit $A$ la matrice avec des $2$ sur la diagonale et des 1 partout ailleurs. Calculer $A^k$ pour tout entier $k$. 

\subsection*{Exercice 3} 

Montrer que le produit de $n$ matrices triangulaires supérieures strictes est nul. 

\clearpage

\section{Groupe symétrique et déterminant} 

\subsection*{Exercice 1} 

Soit $A, B \in \mathcal{M}_n(\R)$ deux matrices qui commutent. Montrer que $\det(A^2 + B^2) \geqslant 0$. 

\subsection*{Exercice 2} 

Soit $\sigma \in \mathfrak{S}_{2n} $ tel que $\sigma(k) = 2k - 1$ et $\sigma(n+k) = 2k \ \forall k \in \n$. Calculer $\varepsilon(\sigma)$. 



\subsection*{Exercice biolley sur une permutation de \{1, ..., 8 \}} 

exos cassini 

\clearpage

\section{Espaces préhilbertiens et espaces euclidiens} 

\subsection*{Questions de cours}

\begin{enumerate}
   \item Démontrer l'inégalité de Cauchy-Schwarz. 
   \item Démontrer le théorème de représentation. 
   \item Montrer que si $(E, (|))$ est de dimension finie et si $F$ est un sev de $E$ alors $E = F\oplus F^{\bot}$. 
\end{enumerate}

\subsection*{Exercice 1 } 

Soit $A\in \R^{m\times n} = \mathcal{M}_{m, n}(\R), b \in \R^m$. On suppose qu'il existe $C>0$ tel que pour tout $x \in \R^n \ \|Ax\| \leqslant C\|x\|$. On pose $f_{\lambda} : x\in \R^n \mapsto \|Ax - b\|_2^2 + \lambda \|x\|_2 \in \R_+$  

Montrer qu'il existe $\lambda_0 > 0$ tel que pour tout $\lambda > \lambda_0 \ \forall x \in \R^n\setminus \{0\} \ f_{\lambda}(x) > \|b\|$.  

indication : Utiliser l'inégalité de Cauchy-Schwartz et l'inégalité triangulaire. 

\subsection*{Exercice 2} 

Calcul de la sous-différentielle de $\iota_C$ pour $C = \left\{ x : \langle w, x \rangle + b = 0\right\}$  

\clearpage

\section{Séries numériques} 

\subsection*{Exercice 1} 

Soit $(p_n)\in \R_+^{\N}$ tel que $\displaystyle \sum_{n=0}^{+\infty}{p_n} = 1$. La série $ \displaystyle \sum{p_n \ln{1/p_n}}$ est-t-elle convergente ? (\textbf{indication :} penser aux séries de Bertrand) 


critères de D'alembert, cauchy, calcul de sommes 

\clearpage

\section{Sources} 
\begin{itemize}
    \item Exo7 
    \item Bibmaths 
    \item Calculus 
    \item Exercices de colles de Thomas Budzinski : https://perso.ens-lyon.fr/thomas.budzinski/colles.pdf 
    \item The Cauchy-Schwartz masterclass 
    \item Exercices de mathématiques oraux x-ens algèbre 1, cassini 
    \item Mathraining 

\end{itemize}
\clearpage

TODO : 

trouver et écrire les énoncés pour les chapitres 10, 11.2, 15, 16, 17, 18, 19. 

\end{document}
